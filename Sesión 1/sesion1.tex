% --------------------------------------------------
%  TALLER DE INTRODUCCIÓN A LaTeX
%  https://github.com/mianfg/latex-intro
%
%  Sesión 1 -> Presentación
%
%  Autor: Miguel Ángel Fernández Gutiérrez, @mianfg
%  Fecha: 20 febrero, 2019
% --------------------------------------------------

% Tipo de documento (presentación)
\documentclass[10pt, xcolor=table]{beamer}

% Cargar el tema
\usetheme{metropolis}

%  __________
% |          |
% | Paquetes |
% |__________|

% Paquetes de idioma
\usepackage[utf8]{inputenc}
\usepackage[spanish, es-tabla, es-lcroman, es-noquoting]{babel}

% Paquete para código fuente
\usepackage{listings}

% Paquete de numeración en Beamer
\usepackage{appendixnumberbeamer}

% Paquete de uso para plantilla
\usepackage{booktabs}
\usepackage[scale=2]{ccicons}

% Paquete para controlar espacios
\usepackage{xspace}
\newcommand{\themename}{\textbf{\textsc{metropolis}}\xspace}

% Paquetes para matemáticas
\usepackage{amsmath}    % Paquete básico de matemáticas
\usepackage{amsthm}     % Teoremas
\usepackage{mathrsfs}   % Fuente para ciertas letras utilizadas en matemáticas

% Paquetes para fuentes
\usepackage{newpxtext, newpxmath}   % Fuente similar a Palatino
\usepackage{FiraSans}               % Fuente sans serif
\usepackage[T1]{fontenc}
\usepackage[italic]{mathastext}     % Utiliza la fuente del documento
                                    % en los entornos matemáticos

%  ________________________
% |                        |
% | Configuración del tema |
% |________________________|

% Configuración básica del tema
\metroset{
  % tema oscuro ('dark') o claro ('light'). No tiene efecto al usar la
  % paleta de colores más adelante
  background=light,
  % 'none' para eliminar la diapositiva inicial de cada sección
  sectionpage=progressbar,
  % 'progressbar' o 'simple' para añadir una diapositiva inicial a cada subsección
  subsectionpage=none,
  % contador de página: 'none', 'counter' o 'fraction'
  numbering=none,
  % barra de progreso: 'none', 'head', 'frametitle' o 'foot'
  progressbar=frametitle,
  % fondo de los bloques estilo teorema: 'transparent' o 'fill'
  block=fill,
}

% Paleta de colores
\definecolor{accent}{RGB}{151, 186, 88}
\colorlet{darkaccent}{accent!70!black}
\definecolor{foreground}{RGB}{0, 0, 0}
\definecolor{background}{RGB}{255, 255, 255}

% Insertar los colores en el tema
\setbeamercolor{normal text}{fg=foreground, bg=background}
\setbeamercolor{alerted text}{fg=darkaccent, bg=background}
\setbeamercolor{example text}{fg=foreground, bg=background}
\setbeamercolor{frametitle}{fg=background, bg=accent}

\setbeamercolor{headtitle}{fg=background!70!accent,bg=accent!90!foreground}
\setbeamercolor{headnav}{fg=background,bg=accent!90!foreground}
\setbeamercolor{section in head/foot}{fg=background,bg=accent}

\defbeamertemplate*{headline}{miniframes theme no subsection}{
  % Caja para mostrar título y autor encima de cada diapositiva
  % Nosotros no 
  %% \begin{beamercolorbox}[ht=2.5ex,dp=1.125ex,
  %%     leftskip=.3cm,rightskip=.3cm plus1fil]{headtitle}
  %%   {\usebeamerfont{title in head/foot}\insertshorttitle}
  %%   \hfill
  %%   \leavevmode{\usebeamerfont{author in head/foot}\insertshortauthor}
  %% \end{beamercolorbox}
  %% \begin{beamercolorbox}[colsep=1.5pt]{upper separation line head}
  %% \end{beamercolorbox}

  % Caja para mostrar navegación encima de cada diapositiva
  \begin{beamercolorbox}{headnav}
    \vskip2pt\insertnavigation{\paperwidth}\vskip2pt
  \end{beamercolorbox}
  \begin{beamercolorbox}[colsep=1.5pt]{lower separation line head}
  \end{beamercolorbox}
}

%  _________
% |         |
% | Ajustes |
% |_________|

% Fijar tabla a posición
\usepackage{array}
\newcolumntype{L}[1]{>{\raggedright\let\newline\\\arraybackslash\hspace{0pt}}m{#1}}
\newcolumntype{C}[1]{>{\centering\let\newline\\\arraybackslash\hspace{0pt}}m{#1}}
\newcolumntype{R}[1]{>{\raggedleft\let\newline\\\arraybackslash\hspace{0pt}}m{#1}}

%  ________
% |        |
% | Título |
% |________|

\title{Primeros pasos con \LaTeX}
\subtitle{Taller de introducción a \LaTeX{}\\ \alert{Sesión 1}}
\date{\today}
\author{Miguel Ángel (@mianfg)}
\institute{\url{https://github.com/mianfg/latex-intro}}
\titlegraphic{\hfill\includegraphics[width=2.5cm]{../assets/mianfg-logo-grey.png}}

%  ___________
% |           |
% | Documento |
% |___________|

\begin{document}

\maketitle

\begin{frame}{Contenidos}
	\setbeamertemplate{section in toc}[sections numbered]
	\tableofcontents[hideallsubsections]
\end{frame}

\section{Qué es \LaTeX{}... y para qué sirve}

\begin{frame}{Qué es \LaTeX{}}
	
	\begin{itemize}
		\item \TeX{} es un programa de ordenador creado por Donald E. Knuth, diseñado para componer texto y fórmulas matemáticas.
		\item \LaTeX{} es un paquete de macros que permite crear contenidos de alta calidad tipográfica, escrito originalmente por Leslie Lamport.
		      \begin{itemize}
		      	\item Emplea \TeX{} como motor de composición.
		      	\item Enfoque \textbf{WYSIWYM} (\emph{What You See Is What You Mean}), en contraposición con el enfoque \textbf{WYSIWYG} (\emph{What You See Is What You Get}) de plataformas como \emph{LibreOffice} o \emph{Microsoft Office}.
		      \end{itemize}
	\end{itemize}
	
\end{frame}
\begin{frame}{Autor, maquetador y compositor}
	\LaTeX{} representa el papel de un \textbf{maquetador}, usando \TeX{} como su compositor. Un maquetador se dedica a interpretar la información proporcionada.
	
	Normalmente, al usar editores WYSIWYG, vemos el resultado final, como con \emph{Word} o \emph{Writer}; este no es el caso de \LaTeX{} en la mayoría de ocasiones.
	
	Cuando se usa \LaTeX{}, no podemos ver el aspecto del fichero hasta que es \textbf{compilado}.
\end{frame}

\begin{frame}{Entonces... ¿para qué quiero \LaTeX{}?}
	\begin{itemize}
		\item Crear documentos formales, especialmente en artículos científicos: TFGs y \emph{papers} se realizan en formatos con \LaTeX{}.
		\item Crear documentos de muy alta calidad.
		\item Crear documentos con contenido matemático: fórmulas, expresiones...
		\item Crear documentos consistentes con mucha funcionalidad: bibliografía, referencias, hiperenlaces...
	\end{itemize}
\end{frame}

\section{Instalación de \LaTeX{}}

\begin{frame}{Instalando \LaTeX{}}
	Dependiendo del sistema operativo, instalaremos:
	\begin{itemize}
		\item Windows
		      \begin{itemize}
		      	\item Usaremos: 
		      \end{itemize}
		\item Linux (o distribuciones de Linux)
		      \begin{itemize}
		      	\item Instalar los siguientes paquetes (en Ubuntu):
		      	\item[] \texttt{sudo apt install texlivefull}
		      	\item[] \texttt{sudo apt install ...}
		      	\item Para otras distros preguntar.
		      \end{itemize}
		      
		\item MacOS
	\end{itemize}
\end{frame}

\begin{frame}[standout]
	Ahora, familiaricémonos con el entorno
\end{frame}

\section{Recursos de este taller}

\begin{frame}{Repositorio de GitHub}
	Podrás encontrar todos los recursos que usaremos en el taller: ejemplos, tutoriales... e incluso estas diapositivas, en el \textbf{repositorio de GitHub} que he creado.
	\begin{center}
		\url{https://github.com/mianfg/latex-intro}
	\end{center}
\end{frame}

\begin{frame}{Más recursos}
	Otros enlaces recomendados para aprender o profundizar \LaTeX{}:
	\begin{itemize}
		\item \LaTeX{} Project: \url{https://www.latex-project.org}
		\item Overleaf Learn: \url{https://www.overleaf.com/learn}
		\item LibreIM (Doble Grado Informática y Matemáticas, UGR): \url{https://libreim.github.io/blog/2015/03/14/latex}
		\item LaTeX-Tutorial: \url{https://www.latex-tutorial.com}
		\item TeX StackExchange: \url{https://tex.stackexchange.com}
		\item Hackr.io: \url{https://hackr.io/tutorials/learn-latex}
	\end{itemize}
	
	
	
\end{frame}

\section{Primeros ejemplos: ficheros de entrada \LaTeX{}}

\begin{frame}{Ficheros de entrada \LaTeX{}}
	Los ficheros de \LaTeX{} son \textbf{archivos de texto plano} que deben ser compilados. Suelen tener la extensión \texttt{.tex}.
	\vspace{0.5cm}
	\begin{alertblock}{Importante}
		Un fichero \LaTeX{} no es lo mismo que el fichero compilado en cualquier formato. Primero modificamos el fichero \texttt{.tex}, y luego lo compilamos, por ejemplo, en un \texttt{.pdf} (además de en otros archivos intermedios).
	\end{alertblock}
\end{frame}

\begin{frame}{Espacios}
	\LaTeX{} trata los caracteres ``en blanco'', tal como el espacio o el tabulador, uniformemente como ``espacios''. \emph{Varios espacios en blanco o tabulaciones consecutivos se consideran como un espacio.}
	
	Una línea vacía entre dos líneas de texto define el fin de un párrafo. \emph{Varias líneas vacías se interpretan como una sola.}
	
	\begin{table}[H]
		\begin{tabular}{ L{4cm} L{0.5cm} L{4cm} }
			\alert{Código \LaTeX{}}                                                                                                                             & \hspace{1cm} & \alert{Resultado}                                                                                 \\
			\cline{1-1}\cline{3-3}
			\cellcolor[RGB]{203,220,171}\texttt{No importa\ \ dejar uno}\newline\texttt{o varios\ \ \ \ espacios.}\newline[0.5cm] \texttt{Igual con los saltos.} &              & \cellcolor[RGB]{203,220,171}No importa dejar uno o varios espacios.\newline Igual con los saltos. \\
		\end{tabular}
	\end{table}
	
\end{frame}

\begin{frame}{Caracteres especiales}
	Estos son caracteres especiales, que se usan en la sintaxis de \LaTeX{}:
	
	\begin{table}[H]
		\begin{tabular}{llllllllll}
			\texttt{\#} & \texttt{\&} & \texttt{\%} & \texttt{\^{}} & \texttt{\&} & \texttt{\_} & \texttt{\~{}} & \texttt{\{} & \texttt{\}} & \texttt{\textbackslash} \\
		\end{tabular}
	\end{table}
	
	Por tanto, para usar estos caracteres debemos \emph{escaparlos}:
	
	\begin{table}[H]
		\begin{tabular}{cccccccccc}
			\texttt{\textbackslash\#} & \texttt{\textbackslash\&} & \texttt{\textbackslash\%} & \cellcolor[RGB]{203,220,171}\texttt{\textbackslash\^{}\{\}} & \texttt{\textbackslash\&} & \texttt{\textbackslash\_} & \cellcolor[RGB]{203,220,171}\texttt{\textbackslash\~{}\{\}} & \texttt{\textbackslash\{} & \texttt{\textbackslash\}} & \cellcolor[RGB]{203,220,171}\texttt{\textbackslash backslash} \\
			                          &                           &                           &                                                             &                           &                           &                                                             &                           &                           & \scriptsize{\texttt{\textbackslash textbackslash}}            
		\end{tabular}
	\end{table}
	
	\begin{alertblock}{¡Cuidado!}
		Para escapar caracteres especiales simplemente insertamos una barra invertida (\texttt{\textbackslash}), \textbf{excepto} en el caso de \texttt{\^{}}, \texttt{\~{}} y el propio \texttt{\textbackslash}.
	\end{alertblock}
	
\end{frame}

\begin{frame}{Órdenes \LaTeX{}}
	Son \textbf{sensibles a mayúsculas}, adoptan uno de los siguientes formatos:
	
	\begin{itemize}
		\item Comienzan con una barra invertida \(\textbackslash\) y tienen a continuación un nombre que consiste únicamente en letras (terminan con un espacio, un número o cualquier cosa que no sea una letra).
		\item Una barra invertida \(\textbackslash\) y exactamente una no-letra.
	\end{itemize}
	
	\begin{table}[H]
		\begin{tabular}{ L{4cm} L{0.5cm} L{4cm} }
			\alert{Código \LaTeX{}}                                                                                                                                                                                                                               & \hspace{1cm} & \alert{Resultado}                                                                                                                             \\
			\cline{1-1}\cline{3-3}
			\cellcolor[RGB]{203,220,171}\texttt{He leído que Knuth divide a la gente que trabaja con \textbackslash TeX\{\} en \textbackslash TeX\{\}nicos y \textbackslash TeX pertos.\textbackslash\textbackslash}\newline\texttt{Hoy es \textbackslash today.} &              & \cellcolor[RGB]{203,220,171}He leído que Knuth divide a la gente que trabaja con \TeX{} en \TeX{}nicos y \TeX pertos.\newline Hoy es \today. \\
		\end{tabular}
	\end{table}
	
\end{frame}

\begin{frame}{Órdenes \LaTeX{}}
	\begin{table}[H]
		\begin{tabular}{ L{4cm} L{0.5cm} L{4cm} }
			\alert{Código \LaTeX{}}                                                                                                                                                                                                                               & \hspace{1cm} & \alert{Resultado}                                                                                                                             \\
			\cline{1-1}\cline{3-3}
			\cellcolor[RGB]{203,220,171}\texttt{He leído que Knuth divide a la gente que trabaja con \textbackslash TeX\{\} en \textbackslash TeX\{\}nicos y \textbackslash TeX pertos.\textbackslash\textbackslash}\newline\texttt{Hoy es \textbackslash today.} &              & \cellcolor[RGB]{203,220,171}He leído que Knuth divide a la gente que trabaja con \TeX{} en \TeX{}nicos y \TeX pertos.\newline Hoy es \today. \\
		\end{tabular}
	\end{table}
	
	\textbf{Cosas que considerar:}
	\begin{itemize}
		\item \LaTeX prescinde del uso del espacio en blanco tras las órdenes. Si quieres conseguir un espacio en blanco tras una orden, pon un \texttt{\{\}} o una orden especial de espaciado tras la orden. Digamos que las llaves impiden a \LaTeX ``comerse'' todo ese espacio.
	\end{itemize}
\end{frame}

\begin{frame}{Órdenes \LaTeX{}}
	\begin{table}[H]
		\begin{tabular}{ L{4cm} L{0.5cm} L{4cm} }
			\alert{Código \LaTeX{}}                                                                                                                                               & \hspace{1cm} & \alert{Resultado}                                                                              \\
			\cline{1-1}\cline{3-3}
			\cellcolor[RGB]{203,220,171}\texttt{Este \textbackslash emph\{magnífico\} texto.\textbackslash newline}\newline\texttt{¡Y \textbackslash textbf\{no olvides\} este!} &              & \cellcolor[RGB]{203,220,171}Este \emph{magnífico} texto.\newline¡Y \textbf{no olvides} este! \\
		\end{tabular}
	\end{table}
	
	\textbf{Cosas que considerar:}
	\begin{itemize}
		\item Algunas órdenes requieren o aceptan \textbf{parámetros} entre las llaves \texttt{\{ \}} tras el nombre de la orden.
		\item Algunas órdenes aceptan \textbf{parámetros opcionales}, que se colocan entre corchetes \texttt{[ ]} tras el nombre de la orden.
	\end{itemize}
\end{frame}

\begin{frame}{Comentarios y el carácter \texttt{\%}}
	Si queremos insertar un comentario, simplemente colocamos un \texttt{\%} al principio de la línea. El compilador ignorará \emph{toda la línea}.
	
	El \texttt{\%} también puede usarse para dividir líneas largas en la entrada donde no se permitan saltos de línea.
	
	\begin{table}[H]
		\begin{tabular}{ L{4cm} L{0.5cm} L{4cm} }
			\alert{Código \LaTeX{}}                                                                                                                                                                                            & \hspace{1cm} & \alert{Resultado}                                                               \\
			\cline{1-1}\cline{3-3}
			\cellcolor[RGB]{203,220,171}\texttt{Este es un \% estúpido}\newline\texttt{\% Mejor->instructivo}\newline\texttt{ejemplo: Excelen\%}\newline\texttt{\ \ \ \ \ \ tísimamente ma\%}\newline\texttt{\ \ \ ravilloso} &              & \cellcolor[RGB]{203,220,171}Este es un ejemplo: excelentísimamente maravilloso \\
		\end{tabular}
	\end{table}
	
\end{frame}

\vspace*{2cm}
Esta presentación ha sido publicada con una licencia Reconocimiento-CompartirIgual 4.0 Internacional (CC BY-SA 4.0).

\begin{center}\ccbysa\end{center}
 
\vspace*{2cm}
\begin{flushright}
	\includegraphics[width=2cm]{../assets/mianfg-logo-grey.png}
\end{flushright}

\section*{}

\end{document}

