\documentclass[11pt, a4paper]{article}

% --------------------------------------------------
%  TALLER DE INTRODUCCIÓN A LaTeX
%  https://github.com/mianfg/latex-intro
%
%  Sesión 1 -> Documento 1
%
%  Autor: Miguel Ángel Fernández Gutiérrez, @mianfg
%  Fecha: 20 febrero, 2019
% --------------------------------------------------

% Paquetes
\usepackage[utf8]{inputenc}				% Codificación
\usepackage[spanish, es-tabla]{babel}	% Idioma
% ...  ¡Iremos ampliando la lista de paquetes!

% Información para \maketitle
\title{\textbf{Primer ejemplo de \LaTeX{}}}
\author{Miguel Ángel Fernández Gutiérrez (@mianfg)}
\date{\today}

% Documento per-se
\begin{document}

\maketitle

\section{Este es un documento básico}

¡Hola! Aquí tienes tu primer documento a \LaTeX{}. Puede ser que estes viendo el \emph{resultado final} o el \emph{código}. En cualquier caso, ¡te felicito!. \textbf{¡Has logrado compilar tu primer documento \LaTeX{}!}

\section{Qué hacer ahora}

Hay varias cosas que puedes hacer:

\begin{itemize}
	\item Es posible que no comprendas muchas partes del código, ¡no te preocupes! \textbf{A lo largo del taller serás capaz de entenderlo todo!}
	\item Por ahora, intenta entender qué hace cada cosa. Así nos acostumbraremos a la metodología de \LaTeX{} de \textbf{texto plano compilado}.
\end{itemize}

\section{¡Prueba tú!}

\textbf{¡Puedes modificar este documento como quieras!}

\end{document}
